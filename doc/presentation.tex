\documentclass[hyperref={colorlinks=true}]{beamer}

\usepackage{fancyvrb,relsize}
\usepackage{graphicx}

\setbeamertemplate{navigation symbols}{}

%\usetheme{Boadilla}
%\usetheme{CambridgeUS}
%\usetheme{Malmoe}
%\usetheme{Singapore}
%\usetheme{boxes}

%\usecolortheme{crane}
%\usecolortheme{dove}
\usecolortheme{seagull} % very cool with \usetheme{default}
%\usefonttheme{professionalfonts}
%\useinnertheme{rectangles}

\mode<presentation>
\title{MK-CONFIGURE -- lightweight easy to use replacement for GNU Autotools}
\author{Aleksey Cheusov \\ \texttt{vle@gmx.net}}
\date{Minsk, Belarus, 2010}

\begin{document}

%%%%%%%%%%%%%%%%%%%%%%%%%%%%%%%%%%%%%%%%%%%%%%%%%%%%%%%%%%%%%%%%%%%%%%

\newenvironment{Code}[1]%
               {\Verbatim[label=\bf{#1},frame=single,%
                   fontsize=\small,%
                   commandchars=\\\{\}]}%
               {\endVerbatim}

\newenvironment{CodeNoLabel}%
               {\Verbatim[frame=single,%
                   fontsize=\small,%
                   commandchars=\\\{\}]}%
               {\endVerbatim}

\newenvironment{CodeNoLabelSmallest}%
               {\Verbatim[frame=single,%
                   fontsize=\footnotesize,%
                   commandchars=\\\{\}]}%
               {\endVerbatim}

%\newcommand{\prompt}[1]{\textcolor{blue}{#1}}
%\newcommand{\prompt}[1]{\textbf{#1}\textnormal{}}
\newcommand{\prompt}[1]{\bf{#1}\textnormal{}}
\newcommand{\highlight}[1]{\bf{#1}\textnormal{}}
\newcommand{\URL}[1]{\textbf{#1}}
\newcommand{\AutohellFile}[1]{\textcolor{red}{#1}}
\newcommand{\MKCfile}[1]{\textcolor{green}{#1}}
\newcommand{\ModuleName}[1]{\textbf{#1}\textnormal{}}
\newcommand{\ProgName}[1]{\textbf{#1}\textnormal{}}
\newcommand{\ProjectName}[1]{\textbf{#1}\textnormal{}}
\newcommand{\PackageName}[1]{\textbf{#1}\textnormal{}}
\newcommand{\MKC}[1]{\large\textsf{#1}\textnormal{}\normalsize}

%%%%%%%%%%%%%%%%%%%%%%%%%%%%%%%%%%%%%%%%%%%%%%%%%%%%%%%%%%%%%%%%%%%%%%
%% \begin{frame}
%%   \frametitle{qqq}
%%   \begin{code}{files in the directory}
%%     bla bla bla
%%   \end{code}
%% \end{frame}

%%%%%%%%%%%%%%%%%%%%%%%%%%%%%%%%%%%%%%%%%%%%%%%%%%%%%%%%%%%%%%%%%%%%%%
\begin{frame}
  \titlepage
\end{frame}

%%%%%%%%%%%%%%%%%%%%%%%%%%%%%%%%%%%%%%%%%%%%%%%%%%%%%%%%%%%%%%%%%%%%%%
\begin{frame}
  \frametitle{Concepts behind mk-configure}
  \begin{block}{Design principles and goals}
    \begin{itemize}
    \item The same way of building projects both for developers and users.
    \item The only file describing the project is(are) Makefile(s).
    \item The only command
      required for building is bmake (portable version of NetBSD make).
    \item Declarative approach of writing Makefile(s). Build and
      installation process is controlled with a help of special
      variables.
    \item No code generation. Library approach is used instead.
    \item No need to ``reinvent'' rules for compiling, linking,
      installing, uninstalling etc. again and again.
    \item KISS. Less than 4000 lines of code.
      No heavy dependencies.
    \end{itemize}
  \end{block}
\end{frame}

%%%%%%%%%%%%%%%%%%%%%%%%%%%%%%%%%%%%%%%%%%%%%%%%%%%%%%%%%%%%%%%%%%%%%%
\begin{frame}
  \frametitle{Concepts behind mk-configure}
  \begin{block}{Design principles and goals}
    \begin{itemize}
    \item Cross-compilation.
    \item Portability to all UNIX-like systems.
    \item Modular approach. Extensions to mk-configure are implemented
      using bmake include files and standard POSIX tools, e.g. shell,
      awk, sed, grep and so on.
    \end{itemize}
  \end{block}
\end{frame}

%%%%%%%%%%%%%%%%%%%%%%%%%%%%%%%%%%%%%%%%%%%%%%%%%%%%%%%%%%%%%%%%%%%%%%
\begin{frame}
  \frametitle{Concepts behind mk-configure}
  \begin{block}{Negative side-effects}
    \begin{itemize}
    \item End-users/packagers have to install bmake and
      mk-configure to build applications based on mk-configure.
    \end{itemize}
  \end{block}
\end{frame}

%%%%%%%%%%%%%%%%%%%%%%%%%%%%%%%%%%%%%%%%%%%%%%%%%%%%%%%%%%%%%%%%%%%%%%
\begin{frame}[fragile]
  \frametitle{Example 1: Hello world application}

  \begin{block}{Source code}
    % files
%%     \begin{code}{files in the directory}
%% \prompt{\$} ls -l
%% total 8
%% -rw-r--r--  1 cheusov  users  68 May  2 14:16 Makefile
%% -rw-r--r--  1 cheusov  users  92 May  2 14:18 hello.c
%% \prompt{\$}
%%     \end{code}

%    \begin{columns}[t]

    % column1
%      \column{0.5\textwidth}
      \begin{Code}{Makefile}
PROG=      hello

.include <mkc.prog.mk>
      \end{Code}

    % column2
%      \column{0.5\textwidth}
      \begin{Code}{hello.c}
#include <stdio.h>

int main (int, char **)
\{
   puts ("Hello World!");
   return 0;
\}
      \end{Code}
%    \end{columns}
  \end{block}
\end{frame}

%%%%%%%%%%%%%%%%%%%%%%%%%%%%%%%%%%%%%%%%%%%%%%%%%%%%%%%%%%%%%%%%%%%%%%
\begin{frame}[fragile]
  \frametitle{Example 1: Hello world application}

%  \begin{block}
\begin{block}{How it works}
\begin{CodeNoLabel}
\prompt{\$ export PREFIX=/usr SYSCONFDIR=/etc}
\prompt{\$ mkcmake}
checking for compiler type... gcc
checking for program cc... /usr/bin/cc
cc     -c hello.c
cc   -o hello hello.o
\prompt{\$ ./hello}
Hello World!
\prompt{\$ DESTDIR=/tmp/fakeroot mkcmake install}
for d in \_ /tmp/fakeroot/usr/bin; do  test "\$d" = \_ ||
   install -d "\$d";  done
install   -c -s  -o cheusov -g users -m 755
   hello /tmp/fakeroot/usr/bin/hello
\prompt{\$}
\end{CodeNoLabel}
\end{block}
Supported targets: all, clean, cleandir (distclean), install,
uninstall, installdirs, depend etc.
%  \end{block}
\end{frame}

%%%%%%%%%%%%%%%%%%%%%%%%%%%%%%%%%%%%%%%%%%%%%%%%%%%%%%%%%%%%%%%%%%%%%%
\begin{frame}[fragile]
  \frametitle{Example 2: Application using non-standard strlcpy(3)}

\begin{block}{Source code}
  % files
  \begin{Code}{files in the directory}
\prompt{\$ ls -l}
total 12
-rw-r--r--  1 cheusov  users  158 May  2 15:04 Makefile
-rw-r--r--  1 cheusov  users  187 May  2 15:05 main.c
-rw-r--r--  1 cheusov  users  332 May  2 15:09 strlcpy.c
\prompt{\$}
  \end{Code}

% Makefile
  \begin{Code}{Makefile}
PROG=                 strlcpy_test
SRCS=                 main.c

MKC_SOURCE_FUNCLIBS=  strlcpy
MKC_CHECK_FUNCS3=     strlcpy:string.h

.include <mkc.prog.mk>
  \end{Code}
\end{block}

\end{frame}

%%%%%%%%%%%%%%%%%%%%%%%%%%%%%%%%%%%%%%%%%%%%%%%%%%%%%%%%%%%%%%%%%%%%%%
\begin{frame}[fragile]
  \frametitle{Example 2: Application using non-standard strlcpy(3)}

\begin{block}{Source code}
\begin{Code}{main.c}
#include <string.h>

#ifndef HAVE_FUNC3_STRLCPY_STRING_H
size_t strlcpy(char *dst, const char *src, size_t siz);
#endif

int main (int argc, char** argv)
\{
    /*    Use strlcpy(3) here    */
    return 0;
\}
\end{Code}
\end{block}
\end{frame}

%%%%%%%%%%%%%%%%%%%%%%%%%%%%%%%%%%%%%%%%%%%%%%%%%%%%%%%%%%%%%%%%%%%%%%
\begin{frame}[fragile]
  \frametitle{Example 2: Application using non-standard strlcpy(3)}

  \begin{block}{How it works on Linux}
\begin{CodeNoLabel}
\prompt{\$ CC='icc -no-gcc' mkcmake}
checking for compiler type... icc
checking for function strlcpy... no
checking for func strlcpy ( string.h )... no
checking for program icc... /opt/intel/cc/10.1.008/bin/icc
icc -no-gcc -c main.c
icc -no-gcc -c strlcpy.c
icc -no-gcc   -o strlcpy_test main.o strlcpy.o
\prompt{\$ echo \_mkc\_*}
_mkc_compiler_type.err _mkc_compiler_type.res 
_mkc_func3_strlcpy_string_h.c 
_mkc_func3_strlcpy_string_h.err 
_mkc_func3_strlcpy_string_h.res 
_mkc_funclib_strlcpy.c _mkc_funclib_strlcpy.err 
_mkc_funclib_strlcpy.res _mkc_prog_cc.err _mkc_prog_cc.res
\prompt{\$}
\end{CodeNoLabel}
  \end{block}
\end{frame}

%%%%%%%%%%%%%%%%%%%%%%%%%%%%%%%%%%%%%%%%%%%%%%%%%%%%%%%%%%%%%%%%%%%%%%
\begin{frame}[fragile]
  \frametitle{Example 2: Application using non-standard strlcpy(3)}

  \begin{block}{How it works on NetBSD}
\begin{CodeNoLabel}
\prompt{\$ mkcmake}
checking for compiler type... gcc
checking for function strlcpy... yes
checking for func strlcpy ( string.h )... yes
checking for program cc... /usr/bin/cc
cc  -DHAVE_FUNC3_STRLCPY_STRING_H=1    -c main.c
cc   -o strlcpy_test main.o
\prompt{\$} 
\end{CodeNoLabel}
  \end{block}
\end{frame}

%%%%%%%%%%%%%%%%%%%%%%%%%%%%%%%%%%%%%%%%%%%%%%%%%%%%%%%%%%%%%%%%%%%%%%
\begin{frame}[fragile]
  \frametitle{Example 3: Application using plugins}

  \begin{block}{Source code}
  \begin{Code}{Makefile}
MKC_CHECK_FUNCLIBS=     dlopen:dl

PROG=                   myapp

.include <mkc.configure.mk>

.if $\{HAVE_FUNCLIB.dlopen\} || $\{HAVE_FUNCLIB.dlopen.dl\}
CFLAGS+=	-DPLUGINS_ENABLED=1
.endif

.include <mkc.prog.mk>
  \end{Code}
  \end{block}
\end{frame}

%%%%%%%%%%%%%%%%%%%%%%%%%%%%%%%%%%%%%%%%%%%%%%%%%%%%%%%%%%%%%%%%%%%%%%
\begin{frame}[fragile]
  \frametitle{Example 3: Application using plugins}

  \begin{block}{How it works on Linux}
\begin{CodeNoLabel}
\prompt{\$ mkcmake}
checking for compiler type... gcc
checking for function dlopen ( \highlight{-ldl} )... \highlight{yes}
checking for function dlopen... \highlight{no}
checking for program gcc... /usr/bin/gcc
gcc -DPLUGINS_ENABLED=1    -c myapp.c
gcc   -o myapp myapp.o \highlight{-ldl}
\prompt{\$}
\end{CodeNoLabel}
  \end{block}
\end{frame}

%%%%%%%%%%%%%%%%%%%%%%%%%%%%%%%%%%%%%%%%%%%%%%%%%%%%%%%%%%%%%%%%%%%%%%
\begin{frame}[fragile]
  \frametitle{Example 3: Application using plugins}

  \begin{block}{How it works on OpenBSD}
\begin{CodeNoLabel}
\prompt{\$ mkcmake}
checking for compiler type... gcc
checking for function dlopen ( \highlight{-ldl} )... \highlight{no}
checking for function dlopen... \highlight{yes}
checking for program cc... /usr/bin/cc
cc  -DPLUGINS_ENABLED=1    -c myapp.c
cc   -o myapp myapp.o
\prompt{\$}
\end{CodeNoLabel}
  \end{block}
\end{frame}

%%%%%%%%%%%%%%%%%%%%%%%%%%%%%%%%%%%%%%%%%%%%%%%%%%%%%%%%%%%%%%%%%%%%%%
\begin{frame}[fragile]
  \frametitle{Example 4: Support for shared libraries}

  \begin{block}{Source code}
  \begin{Code}{Makefile}
LIB=                   foobar
SRCS=                  foo.cc bar.cc baz.cc

MKPICLIB?=             no
MKSTATICLIB?=          no

SHLIB_MAJOR=           1
SHLIB_MINOR=           0

.include <mkc.lib.mk>
  \end{Code}
  \end{block}
\end{frame}

%%%%%%%%%%%%%%%%%%%%%%%%%%%%%%%%%%%%%%%%%%%%%%%%%%%%%%%%%%%%%%%%%%%%%%
\begin{frame}[fragile]
  \frametitle{Example 4: Support for shared libraries}

  \begin{block}{How it works on Solaris}
\begin{CodeNoLabel}
\prompt{\$ mkcmake}
/opt/SUNWspro/bin/CC    -c -KPIC foo.cc -o foo.os
/opt/SUNWspro/bin/CC    -c -KPIC bar.cc -o bar.os
/opt/SUNWspro/bin/CC    -c -KPIC baz.cc -o baz.os
building shared foobar library (version 1.0)
/opt/SUNWspro/bin/CC -G -h libfoobar.so.1 
   -o libfoobar.so.1.0  foo.os bar.os baz.os
ln -sf libfoobar.so.1.0 libfoobar.so
ln -sf libfoobar.so.1.0 libfoobar.so.1
\prompt{\$}
\end{CodeNoLabel}
  \end{block}
\end{frame}

%%%%%%%%%%%%%%%%%%%%%%%%%%%%%%%%%%%%%%%%%%%%%%%%%%%%%%%%%%%%%%%%%%%%%%
\begin{frame}[fragile]
  \frametitle{Example 4: Support for shared libraries}

  \begin{block}{How it works on Darwin}
\begin{CodeNoLabel}
\prompt{\$ mkcmake}
checking for compiler type... gcc
checking for program c++... /usr/bin/c++
c++    -c -fPIC -DPIC foo.cc -o foo.os
c++    -c -fPIC -DPIC bar.cc -o bar.os
c++    -c -fPIC -DPIC baz.cc -o baz.os
building shared foobar library (version 1.0)
c++ -dynamiclib -install_name 
   /usr/local/lib/libfoobar.1.0.dylib 
   -current_version 2.0 -compatibility_version 2 
   -o libfoobar.1.0.dylib  foo.os bar.os baz.os
ln -sf libfoobar.1.0.dylib libfoobar.dylib
ln -sf libfoobar.1.0.dylib libfoobar.1.dylib
\prompt{\$}
\end{CodeNoLabel}
  \end{block}
\end{frame}

%%%%%%%%%%%%%%%%%%%%%%%%%%%%%%%%%%%%%%%%%%%%%%%%%%%%%%%%%%%%%%%%%%%%%%
\begin{frame}[fragile]
  \frametitle{Example 5: Big project consisting of several subprojects}

  \begin{block}{Source code}
  \begin{Code}{Makefile}
# This project consists of several subprojects:
# dict, dictd, dictfmt, dictzip, libdz, libmaa
# and libcommon. libcommon contains common code
# for executables and should not be installed.
# SUBPRJ specifies a dependency graph
# for all subprojects.

SUBPRJ=   libcommon:dict   # dict depends on libcommon
SUBPRJ+=  libcommon:dictd
SUBPRJ+=  libcommon:dictzip
SUBPRJ+=  libcommon:dictfmt
SUBPRJ+=  libdz:dictzip
SUBPRJ+=  libmaa:dict
...
.include <mkc.subprj.mk>
  \end{Code}
  \end{block}
\end{frame}

%%%%%%%%%%%%%%%%%%%%%%%%%%%%%%%%%%%%%%%%%%%%%%%%%%%%%%%%%%%%%%%%%%%%%%
\begin{frame}[fragile]
  \frametitle{Example 5: Big project consisting of several subprojects}

  \begin{block}{Source code}
  \begin{Code}{libcommon/Makefile}
# Internal static library that implements functions
# common for dict, dictd, dictfmt
# and dictzip applications

LIB=            common
SRCS=           str.c iswalnum.c # and others

MKINSTALL=      no # Do not install it!

.include <mkc.lib.mk>
  \end{Code}
  \begin{Code}{libcommon/linkme.mk}
PATH.common:=      \$\{.PARSEDIR\}

CPPFLAGS+=      -I\$\{PATH.common\}
DPLIBDIRS+=     \$\{PATH.common\}
  \end{Code}
  \end{block}
\end{frame}

%%%%%%%%%%%%%%%%%%%%%%%%%%%%%%%%%%%%%%%%%%%%%%%%%%%%%%%%%%%%%%%%%%%%%%
\begin{frame}[fragile]
  \frametitle{Example 5: Big project consisting of several subprojects}

  \begin{block}{Source code}
  \begin{Code}{libmaa/Makefile}
LIB=            maa
SRCS=           set.c prime.c log.c # etc.

SHLIB_MAJOR=    1
SHLIB_MINOR=    2
SHLIB_TEENY=    0

# list of exported symbols
EXPORT_SYMBOLS= libmaa.sym

.include <mkc.lib.mk>
  \end{Code}
  \begin{Code}{libmaa/linkme.mk}
PATH.maa:=      \$\{.PARSEDIR\}

CPPFLAGS+=      -I\$\{PATH.maa\}
DPLIBDIRS+=     \$\{PATH.maa\}
  \end{Code}
  \end{block}
\end{frame}

%%%%%%%%%%%%%%%%%%%%%%%%%%%%%%%%%%%%%%%%%%%%%%%%%%%%%%%%%%%%%%%%%%%%%%
\begin{frame}[fragile]
  \frametitle{Example 5: Big project consisting of several subprojects}

  \begin{block}{Source code}
  \begin{Code}{libdz/Makefile}
LIB=            dz
SRCS=           dz.c

MKC_REQUIRE_HEADERS=    zlib.h
MKC_REQUIRE_FUNCLIBS=   deflate:z

EXPORT_SYMBOLS=         libdz.sym
SHLIB_MAJOR=            1
SHLIB_MINOR=            0

.include <mkc.lib.mk>
  \end{Code}
  \begin{Code}{libdz/linkme.mk}
PATH.dz:=       \$\{.PARSEDIR\}

CPPFLAGS+=      -I\$\{PATH.dz\}
DPLIBDIRS+=     \$\{PATH.dz\}
  \end{Code}
  \end{block}
\end{frame}

%%%%%%%%%%%%%%%%%%%%%%%%%%%%%%%%%%%%%%%%%%%%%%%%%%%%%%%%%%%%%%%%%%%%%%
\begin{frame}[fragile]
  \frametitle{Example 5: Big project consisting of several subprojects}

  \begin{block}{Source code}
  \begin{Code}{dictzip/Makefile}
PROG=   dictzip
MAN=    dictzip.1

.include "../libcommon/linkme.mk"
.include "../libdz/linkme.mk"
.include "../libmaa/linkme.mk"

DPLIBS+=        -lcommon -ldz -lmaa

.include <mkc.prog.mk>
  \end{Code}
  \end{block}
\end{frame}

%%%%%%%%%%%%%%%%%%%%%%%%%%%%%%%%%%%%%%%%%%%%%%%%%%%%%%%%%%%%%%%%%%%%%%
\begin{frame}[fragile]
  \frametitle{Example 5: Big project consisting of several subprojects}

  \begin{block}{How it works}
  \begin{CodeNoLabel}
\prompt{\$ mkcmake dictzip}
==================================================
all ===> libcommon
...
==================================================
all ===> libdz
...
==================================================
all ===> dictzip
...
checking for program cc... /usr/bin/cc
cc   -I../libcommon -I../libdz -I../libmaa  -c dictzip.c
cc -L/tmp/hello_dictd/libcommon -L/tmp/hello_dictd/libdz
   -L/tmp/hello_dictd/libmaa  -o dictzip
   dictzip.o -lcommon -lmaa -ldz
\prompt{\$}
  \end{CodeNoLabel}
  \end{block}
\end{frame}

%%%%%%%%%%%%%%%%%%%%%%%%%%%%%%%%%%%%%%%%%%%%%%%%%%%%%%%%%%%%%%%%%%%%%%
\begin{frame}[fragile]
  \frametitle{Example 6: Support for Lua programming language}

  \begin{block}{Source code}
  \begin{Code}{Makefile}
SCRIPTS=       foobar  # scripts written in Lua
LUA\_LMODULES=  foo bar # modules written in Lua
LUA\_CMODULE=   baz     # Lua module written in C

.include <mkc.lib.mk>
  \end{Code}
  \end{block}
\end{frame}

%%%%%%%%%%%%%%%%%%%%%%%%%%%%%%%%%%%%%%%%%%%%%%%%%%%%%%%%%%%%%%%%%%%%%%
\begin{frame}[fragile]
  \frametitle{Example 6: Support for Lua programming language}

  \begin{block}{How it works}
  \begin{CodeNoLabel}
\prompt{\$ mkcmake errorcheck}
checking for program pkg-config...
   /usr/pkg/bin/pkg-config
checking for [pkg-config] lua... 1 (yes)
checking for [pkg-config] lua --cflags...
   -I/usr/pkg/include
checking for [pkg-config] lua --libs...
   -Wl,-R/usr/pkg/lib -L/usr/pkg/lib -llua -lm
checking for [pkg-config] lua --variable=INSTALL_LMOD...
   /usr/pkg/share/lua/5.1
checking for [pkg-config] lua --variable=INSTALL_CMOD...
   /usr/pkg/lib/lua/5.1
checking for compiler type... gcc
checking for header lua.h... yes
checking for program cc... /usr/bin/cc
\prompt{\$}
  \end{CodeNoLabel}
  \end{block}
\end{frame}

%%%%%%%%%%%%%%%%%%%%%%%%%%%%%%%%%%%%%%%%%%%%%%%%%%%%%%%%%%%%%%%%%%%%%%
\begin{frame}[fragile]
  \frametitle{Example 6: Support for Lua programming language}

  \begin{block}{How it works}
  \begin{CodeNoLabel}
\prompt{\$ export PREFIX=/usr/pkg}
\prompt{\$ mkcmake all}
cc -DHAVE_HEADER_LUA_H=1 -I/usr/pkg/include
   -c -fPIC -DPIC baz.c -o baz.os
building shared baz library (version 1.0)
cc -shared -Wl,-soname -Wl,libbaz.so.1 -o baz.so  baz.os
   -Wl,-R/usr/pkg/lib -L/usr/pkg/lib -llua -lm
\prompt{\$}
  \end{CodeNoLabel}
  \end{block}
\end{frame}

%%%%%%%%%%%%%%%%%%%%%%%%%%%%%%%%%%%%%%%%%%%%%%%%%%%%%%%%%%%%%%%%%%%%%%
\begin{frame}[fragile]
  \frametitle{Example 6: Support for Lua programming language}

  \begin{block}{How it works}
  \begin{CodeNoLabel}
\prompt{\$ mkcmake install DESTDIR=/tmp/fakeroot}
    ...
\prompt{\$ find /tmp/fakeroot -type f}
/tmp/fakeroot/usr/pkg/bin/foobar
/tmp/fakeroot/usr/pkg/lib/lua/5.1/baz.so
/tmp/fakeroot/usr/pkg/share/lua/5.1/foo.lua
/tmp/fakeroot/usr/pkg/share/lua/5.1/bar.lua
\prompt{\$}
  \end{CodeNoLabel}
  \end{block}
\end{frame}

%%%%%%%%%%%%%%%%%%%%%%%%%%%%%%%%%%%%%%%%%%%%%%%%%%%%%%%%%%%%%%%%%%%%%%
\begin{frame}[fragile]
  \frametitle{Example 7: Portable version of AWK from NetBSD}

\begin{block}{http://mova.org/\~{}cheusov/pub/mk-configure/nbawk/}
\begin{CodeNoLabelSmallest}
PROG=	awk
SRCS=	awkgram.y b.c lex.c lib.c main.c parse.c
        proctab.c run.c tran.c
YHEADER=	yes
MKC\_COMMON\_DEFINES.Linux=	-D\_GNU\_SOURCE
MKC\_COMMON\_HEADERS=		ctype.h stdio.h string.h
MKC\_CHECK\_FUNCS1=		\_\_fpurge:stdio\_ext.h fpurge isblank
MKC\_CHECK\_FUNCS3=		strlcat
MKC\_SOURCE\_FUNCLIBS=		fpurge strlcat
.include <mkc.configure.mk>
.if \$\{HAVE\_FUNC1.isblank:U0\}
CPPFLAGS+=	-DHAS\_ISBLANK
.endif
.if !\$\{HAVE\_FUNC1.fpurge:U1\} && !\$\{HAVE_FUNC1.\_\_fpurge.stdio\_ext\_h:U1\}
MKC\_ERR\_MSG+=	"fpurge(3) cannot be found"
.endif
CPPFLAGS+=	-I.
LDADD+=		-lm
.include <mkc.prog.mk>
\end{CodeNoLabelSmallest}
\end{block}
\end{frame}

%%%%%%%%%%%%%%%%%%%%%%%%%%%%%%%%%%%%%%%%%%%%%%%%%%%%%%%%%%%%%%%%%%%%%%
\begin{frame}[fragile]
  \frametitle{Example 7: Portable version of AWK from NetBSD}

\begin{block}{run.c}
\begin{CodeNoLabelSmallest}
--- nbawk-20100903/run.c.orig
+++ nbawk-20100903/run.c
@@ -40,6 +40,14 @@
 #include "awk.h"
 #include "awkgram.h"
 
+#ifndef HAVE_FUNC1_FPURGE
+int fpurge (FILE *stream);
+#endif
+
+#ifndef HAVE\_FUNC3\_STRLCAT
+size\_t strlcat(char *dst, const char *src, size_t size);
+#endif
+
 #define tempfree(x)    if (istemp(x)) tfree(x); else
 
 void stdinit(void);
\end{CodeNoLabelSmallest}
\end{block}
\end{frame}

%%%%%%%%%%%%%%%%%%%%%%%%%%%%%%%%%%%%%%%%%%%%%%%%%%%%%%%%%%%%%%%%%%%%%%
\begin{frame}[fragile]
  \frametitle{Example 7: Portable version of AWK from NetBSD}

\begin{block}{fpurge.c}
\begin{CodeNoLabelSmallest}
#include <stdio.h>

#if HAVE\_FUNC1\_\_\_FPURGE\_STDIO\_EXT\_H
#include <stdio\_ext.h>
#endif

int fpurge(FILE *stream);

int fpurge(FILE *stream)
\{
#if HAVE\_FUNC1\_\_\_FPURGE\_STDIO\_EXT\_H
  \_\_fpurge (stream);
  return 0;
#else
#error "cannot find fpurge(3), sorry"
#endif
\}
\end{CodeNoLabelSmallest}
\end{block}
\end{frame}

%%%%%%%%%%%%%%%%%%%%%%%%%%%%%%%%%%%%%%%%%%%%%%%%%%%%%%%%%%%%%%%%%%%%%%
\begin{frame}[fragile]
  \frametitle{Example 7: Portable version of AWK from NetBSD}

\begin{block}{strlcpy.c}
\begin{CodeNoLabel}
If you want this code, you know where to get it! ;-)
\end{CodeNoLabel}
\end{block}
\end{frame}

%%%%%%%%%%%%%%%%%%%%%%%%%%%%%%%%%%%%%%%%%%%%%%%%%%%%%%%%%%%%%%%%%%%%%%
\begin{frame}[fragile]
  \frametitle{Features}
  \begin{block}{}
  \begin{enumerate}
  \item Automatic detection of system configuration
    (\ModuleName{mkc.configure.mk})
    \begin{itemize}
    \item header presence (MKC\_\{CHECK,REQUIRE\}\_HEADERS)
    \item function declaration (MKC\_\{CHECK,REQUIRE\}\_FUNCS[n])
    \item type declaration (MKC\_\{CHECK,REQUIRE\}\_TYPES)
    \item structure member (MKC\_\{CHECK,REQUIRE\}\_MEMBERS)
    \item variable declaration (MKC\_\{CHECK,REQUIRE\}\_VARS)
    \item define declaration (MKC\_\{CHECK,REQUIRE\}\_DEFINES)
    \item type size (MKC\_CHECK\_SIZEOF)
    \item function implementation in the library
      (MKC\_\{CHECK,REQUIRE\}\_FUNCLIBS and MKC\_SOURCE\_FUNCLIBS)
    \item checks for programs (MKC\_\{CHECK,REQUIRE\}\_PROGS)
    \item user's custom checks (MKC\_\{CHECK,REQUIRE\}\_CUSTOM)
    \item built-in checks (MKC\_CHECK\_BUILTINS), e.g. endianess,
      prog\_flex, prog\_bison, prog\_gawk or prog\_gm4)
    \end{itemize}
  \end{enumerate}
  \end{block}
\end{frame}

%%%%%%%%%%%%%%%%%%%%%%%%%%%%%%%%%%%%%%%%%%%%%%%%%%%%%%%%%%%%%%%%%%%%%%
\begin{frame}[fragile,t]
  \frametitle{Features}
  \begin{block}{}
  \begin{enumerate}
  \setcounter{enumi}{1}
  \item Building, installing, uninstalling, cleaning
    etc. Supported targets: all, installdirs, install, uninstall,
    clean, cleandir (distclean) and others.
  \item Building standalone programs (\ModuleName{mkc.prog.mk}),
    static and shared libraries (\ModuleName{mkc.lib.mk}) using C,
    C++, Objective C, Pascal and Fortran compilers. Shared libraries
    support is provided for numerous OSes: NetBSD, FreeBSD, OpenBSD,
    DragonFlyBSD, MirOS BSD, Linux, Solaris, Darwin (MacOS-X), Tru64,
    QNX, HP-UX, Cygwin (no support for shared object files yet) and
    compilers: GCC, Intel C/C++ compilers, Portable C compiler aka
    pcc, DEC C/C++ compiler, HP C/C++ compiler, Oracle SunPro and others.
  \item Handling of man pages.
  \end{enumerate}
  \end{block}
\end{frame}

%%%%%%%%%%%%%%%%%%%%%%%%%%%%%%%%%%%%%%%%%%%%%%%%%%%%%%%%%%%%%%%%%%%%%%

\begin{frame}[fragile,t]
  \frametitle{Features}
  \begin{block}{}
  \begin{enumerate}
  \setcounter{enumi}{4}
  \item Building info pages from texinfo sources.
  \item Handling of scripts as well as plain text files,
    i.e. installing or uninstalling.
  \item Cross-building. mk-configure itself doesn't run produced
    executables, so you can freely use cross-tools (compiler, linker
    etc.).  Also you can override any variable initialized by mk-configure.
  \item Support for pkg-config
    (\ModuleName{mkc\_imp.pkg-config.mk}).
  \item Support for Lua programming language
    (\ModuleName{mkc\_imp.lua.mk}).
  \item Support for yacc and lex.
  \item Support for projects with multiple subprojects
    (\ModuleName{mkc.subprj.mk} and \ModuleName{mkc.subdir.mk}).
  \end{enumerate}
  \end{block}
\end{frame}

%%%%%%%%%%%%%%%%%%%%%%%%%%%%%%%%%%%%%%%%%%%%%%%%%%%%%%%%%%%%%%%%%%%%%%

\begin{frame}[fragile,t]
  \frametitle{Features}
  \begin{block}{}
  \begin{enumerate}
    \setcounter{enumi}{11}
    \item Special targets bin\_tar, bin\_targz, bin\_tarbz2, bin\_zip,
      bin\_deb creates .tar, .tar.gz, .tar.bz2, .zip archives and .deb
      package (on Debian Linux).
    \item Parts of mk-configure functionality is
      accesible via individual programs, e.g.  \ProgName{mkc\_install},
      \ProgName{mkc\_check\_compiler},
      \ProgName{mkc\_check\_header}, \ProgName{mkc\_check\_funclib},
      \ProgName{mkc\_check\_decl},
      \ProgName{mkc\_check\_prog}, \ProgName{mkc\_check\_sizeof} and
      \ProgName{mkc\_check\_custom}.
  \end{enumerate}
  \end{block}
\end{frame}

%%%%%%%%%%%%%%%%%%%%%%%%%%%%%%%%%%%%%%%%%%%%%%%%%%%%%%%%%%%%%%%%%%%%%%

\begin{frame}[fragile,t]
  \frametitle{MK-CONFIGURE in real world}
  \begin{block}{Packages in UNIX-like system and distributions}
    NetBSD make (bmake) is packaged in the following OSes:
    \begin{itemize}
    \item FreeBSD, NetBSD
    \item Gentoo Linux, Fedora Linux, AltLinux
    \item Debian/Ubuntu\\
      deb http://mova.org/\~{}cheusov/pub/debian lenny main
      deb-src http://mova.org/\~{}cheusov/pub/debian lenny main
    \end{itemize}
    mk-configure is packaged in the following OSes
    \begin{itemize}
    \item FreeBSD, NetBSD
    \item AltLinux
    \item Debian/Ubuntu\\
      deb http://mova.org/\~{}cheusov/pub/debian lenny main
      deb-src http://mova.org/\~{}cheusov/pub/debian lenny main
    \end{itemize}
  \end{block}
\end{frame}

%%%%%%%%%%%%%%%%%%%%%%%%%%%%%%%%%%%%%%%%%%%%%%%%%%%%%%%%%%%%%%%%%%%%%%

\begin{frame}[fragile,t]
  \frametitle{MK-CONFIGURE in real world}
  \begin{block}{Real life samples of use}
  \begin{itemize}
  \item Lightweight modular malloc Debugger.\\
    \URL{http://sf.net/projects/lmdbg/}
  \item NetBSD version of AWK programming language.\\
    \URL{http://mova.org/~cheusov/pub/mk-configure/nbawk/}
  \item Any project based on traditional
    \ModuleName{bsd.\{prog,lib,subdir\}.mk} can easily be converted
    to use \MKC{mk-configure}.
    \begin{itemize}
    \item \URL{http://sourceforge.net/projects/runawk/}\\
      Modules Framework for AWK programming language
    \item \URL{http://sourceforge.net/projects/paexec/}\\
      Parallel Executer
    \item \URL{http://mova.org/\~{}cheusov/pub/distbb/}\\
      Distributed fault tolerant bulk build tool for pkgsrc
      \URL{http://mova.org/\~{}cheusov/pub/pkg\_online/}\\
      Client/server package search system for pkgsrc
    \item ...
    \end{itemize}
  \end{itemize}
  \end{block}
\end{frame}

%%%%%%%%%%%%%%%%%%%%%%%%%%%%%%%%%%%%%%%%%%%%%%%%%%%%%%%%%%%%%%%%%%%%%%

\begin{frame}[fragile,t]
  \frametitle{MK-CONFIGURE in real world}
  \begin{block}{My opensource software projects using
      mk-configure (romb), Mk files (box) and others (oval)}
    \begin{figure}
      \includegraphics[width=\textwidth, keepaspectratio=false]{my_prjs.eps}
    \end{figure}
    \begin{figure}
      \includegraphics[width=0.9\textwidth, keepaspectratio=false]{my_prjs2.eps}
    \end{figure}
  \end{block}
\end{frame}

%%%%%%%%%%%%%%%%%%%%%%%%%%%%%%%%%%%%%%%%%%%%%%%%%%%%%%%%%%%%%%%%%%%%%%

\end{document}
