%%%begin-myprojects
\documentclass[hyperref={colorlinks=true}]{beamer}

\usepackage{fancyvrb,relsize}
\usepackage{graphicx}

\setbeamertemplate{navigation symbols}{}

%\usetheme{Boadilla}
%\usetheme{CambridgeUS}
%\usetheme{Malmoe}
%\usetheme{Singapore}
%\usetheme{boxes}

%\usecolortheme{crane}
%\usecolortheme{dove}
\usecolortheme{seagull} % very cool with \usetheme{default}
%\usefonttheme{professionalfonts}
%\useinnertheme{rectangles}

\mode<presentation>
\title{MK-CONFIGURE (MK-C) -- lightweight,
  easy to use alternative for GNU Autotools}
\author{Aleksey Cheusov \\ \texttt{vle@gmx.net}}
\date{Minsk, Belarus, 2011}

\begin{document}

%%%%%%%%%%%%%%%%%%%%%%%%%%%%%%%%%%%%%%%%%%%%%%%%%%%%%%%%%%%%%%%%%%%%%%

\newenvironment{Code}[1]%
              {\Verbatim[label=\bf{#1},frame=single,%
                  fontsize=\small,%
                  commandchars=\\\{\}]}%
              {\endVerbatim}

%\newenvironment{Code}[1]%
%               {\semiverbatim[]}%
%               {\endsemiverbatim}

\newenvironment{CodeNoLabel}%
               {\Verbatim[frame=single,%
                   fontsize=\small,%
                   commandchars=\\\{\}]}%
               {\endVerbatim}

\newenvironment{CodeNoLabelSmallest}%
               {\Verbatim[frame=single,%
                   fontsize=\footnotesize,%
                   commandchars=\\\{\}]}%
               {\endVerbatim}
\newenvironment{CodeLarge}%
               {\Verbatim[frame=single,%
                   fontsize=\large,%
                   commandchars=\\\{\}]}%
               {\endVerbatim}

%\newcommand{\prompt}[1]{\textcolor{blue}{#1}}
%\newcommand{\prompt}[1]{\textbf{#1}\textnormal{}}
\newcommand{\prompt}[1]{{\bf{#1}}}
%\newcommand{\h}[1]{\textbf{#1}}
%\newcommand{\h}[1]{\bf{#1}\textnormal{}}
\newcommand{\h}[1]{{\bf{#1}}}
\newcommand{\URL}[1]{\textbf{#1}}
\newcommand{\AutohellFile}[1]{\textcolor{red}{#1}}
\newcommand{\MKCfile}[1]{\textcolor{green}{#1}}
\newcommand{\ModuleName}[1]{\textbf{#1}\textnormal{}}
\newcommand{\ProgName}[1]{\textbf{#1}\textnormal{}}
\newcommand{\ProjectName}[1]{\textbf{#1}\textnormal{}}
\newcommand{\PackageName}[1]{\textbf{#1}\textnormal{}}
\newcommand{\MKC}[1]{\large\textsf{#1}\textnormal{}\normalsize}

%%%%%%%%%%%%%%%%%%%%%%%%%%%%%%%%%%%%%%%%%%%%%%%%%%%%%%%%%%%%%%%%%%%%%%
%% \begin{frame}
%%   \frametitle{qqq}
%%   \begin{code}{files in the directory}
%%     bla bla bla
%%   \end{code}
%% \end{frame}

%%%end-myprojects

%%%%%%%%%%%%%%%%%%%%%%%%%%%%%%%%%%%%%%%%%%%%%%%%%%%%%%%%%%%%%%%%%%%%%%
\begin{frame}
  \titlepage
\end{frame}

%%%%%%%%%%%%%%%%%%%%%%%%%%%%%%%%%%%%%%%%%%%%%%%%%%%%%%%%%%%%%%%%%%%%%%
\begin{frame}
  \frametitle{About this presentation}
  \begin{block}{}
    \begin{itemize}
    \item It is a part of official documentation.
      Latest version is available for download from here\\
      \URL{http://mova.org/\~{}cheusov/pub/mk-c/mk-c.pdf}
%    \item
%      \URL{http://sourceforge.net/projects/mk-configure} \\
%      \URL{http://freshmeat.net/projects/mk-configure}
    \item part 1: Introduction
    \item part 2: A number of samples of use
    \item part 3: More complete list of features, TODO and more.
    \end{itemize}
  \end{block}
\end{frame}

%%%%%%%%%%%%%%%%%%%%%%%%%%%%%%%%%%%%%%%%%%%%%%%%%%%%%%%%%%%%%%%%%%%%%%
\begin{frame}
  \frametitle{Concepts behind mk-configure}
  \begin{block}{Design principles and goals}
    \begin{itemize}
    \item \h{I detest code generation} as in Autotools and CMake!\\
      \h{Library approach} is used instead.
    \item Written in \h{bmake}, portable version of \h{NetBSD make(1)},
      and UNIX tools. \h{No heavy dependencies} like python, ruby and perl.
      As a programming language
      bmake is not as powerful as RuPyPe, but\\
      \h{\mbox{bmake+sh} is good enough} for this task.
    \item Basic principles are \h{similar to
      traditional \mbox{bsd.*.mk} files}.
      Actually mk-c contains heavily reworked Mk files from NetBSD.
    \item \h{Portability} to all UNIX-like systems.
    \item KISS.  Only about 4000 lines of code.
    \end{itemize}
  \end{block}
\end{frame}

%%%%%%%%%%%%%%%%%%%%%%%%%%%%%%%%%%%%%%%%%%%%%%%%%%%%%%%%%%%%%%%%%%%%%%
\begin{frame}
  \frametitle{Concepts behind mk-configure}
  \begin{block}{Design principles and goals}
    \begin{itemize}
    \item mk-configure is not only for end-users and packagers
      but for developers too.
      So, one of the main goals is to provide a convenient \h{tool for
        development}.
    \item Declarative approach of writing Makefile(s). Build and
      installation process is controlled with a help of special
      variables and bmake's include files.
    \item \h{Cross-compilation}.
    \item \h{Extensibility}. Extensions to mk-configure are implemented
      using bmake include files and standard UNIX tools, i.e. shell,
      awk, sed, grep etc. when needed.
    \item MK-C is \h{Easy to use}. Only one command is needed to build
      a project --- \h{mkcmake}.  \h{Only Makefile(s)} are needed for
      specifying build instructions.
    \end{itemize}
  \end{block}
\end{frame}

%%%%%%%%%%%%%%%%%%%%%%%%%%%%%%%%%%%%%%%%%%%%%%%%%%%%%%%%%%%%%%%%%%%%%%
\begin{frame}
  \frametitle{Concepts behind mk-configure}
  \begin{block}{Negative side-effects}
    \begin{itemize}
    \item End-users/packagers have to install bmake and
      mk-configure to build applications based on mk-configure.
    \end{itemize}
  \end{block}
\end{frame}

%%%%%%%%%%%%%%%%%%%%%%%%%%%%%%%%%%%%%%%%%%%%%%%%%%%%%%%%%%%%%%%%%%%%%%
\begin{frame}[fragile]
  \frametitle{Example 1: Hello world application}

  \begin{block}{Source code}
      \begin{Code}{Makefile}
PROG =      hello

.include <\h{mkc.prog.mk}>
      \end{Code}
      \begin{Code}{hello.c}
#include <stdio.h>

int main (int, char **)
\{
   puts ("Hello World!");
   return 0;
\}
      \end{Code}
%    \end{columns}
  \end{block}
\end{frame}

%%%%%%%%%%%%%%%%%%%%%%%%%%%%%%%%%%%%%%%%%%%%%%%%%%%%%%%%%%%%%%%%%%%%%%
\begin{frame}[fragile]
  \frametitle{Example 1: Hello world application}

%  \begin{block}
\begin{block}{How it works}
\begin{CodeNoLabel}
\prompt{\$ export PREFIX=/usr SYSCONFDIR=/etc}
\prompt{\$ mkcmake}
checking for compiler type... gcc
checking for program cc... /usr/bin/cc
cc     -c hello.c
cc   -o hello hello.o
\prompt{\$ ./hello}
Hello World!
\prompt{\$ DESTDIR=/tmp/fakeroot mkcmake install}
for d in \_ /tmp/fakeroot/usr/bin; do  test "\$d" = \_ ||
   install -d "\$d";  done
install   -c -s  -o cheusov -g users -m 755
   hello /tmp/fakeroot/usr/bin/hello
\prompt{\$}
\end{CodeNoLabel}
\end{block}
Supported targets: all, clean, cleandir (distclean), install,
uninstall, installdirs, depend etc.
%  \end{block}
\end{frame}

%%%%%%%%%%%%%%%%%%%%%%%%%%%%%%%%%%%%%%%%%%%%%%%%%%%%%%%%%%%%%%%%%%%%%%
\begin{frame}[fragile]
  \frametitle{Example 2: Application using non-standard strlcpy(3)}

\begin{block}{Source code}
  % files
  \begin{Code}{files in the directory}
\prompt{\$ ls -l}
total 12
-rw-r--r--  1 cheusov  users  158 May  2 15:04 Makefile
-rw-r--r--  1 cheusov  users  187 May  2 15:05 main.c
-rw-r--r--  1 cheusov  users  332 May  2 15:09 \h{strlcpy.c}
\prompt{\$}
  \end{Code}

% Makefile
  \begin{Code}{Makefile}
PROG =                 strlcpy_test
SRCS =                 main.c

\h{MKC\_SOURCE\_FUNCLIBS} =  strlcpy
\h{MKC\_CHECK\_FUNCS3} =     strlcpy:string.h

.include <mkc.prog.mk>
  \end{Code}
\end{block}

\end{frame}

%%%%%%%%%%%%%%%%%%%%%%%%%%%%%%%%%%%%%%%%%%%%%%%%%%%%%%%%%%%%%%%%%%%%%%
\begin{frame}[fragile]
  \frametitle{Example 2: Application using non-standard strlcpy(3)}

\begin{block}{Source code}
\begin{Code}{main.c}
#include <string.h>

#ifndef \h{HAVE\_FUNC3\_STRLCPY\_STRING\_H}
size_t strlcpy(char *dst, const char *src, size_t siz);
#endif

int main (int argc, char** argv)
\{
    /*    Use strlcpy(3) here    */
    return 0;
\}
\end{Code}
\end{block}
\end{frame}

%%%%%%%%%%%%%%%%%%%%%%%%%%%%%%%%%%%%%%%%%%%%%%%%%%%%%%%%%%%%%%%%%%%%%%
\begin{frame}[fragile]
  \frametitle{Example 2: Application using non-standard strlcpy(3)}

  \begin{block}{How it works on Linux}
\begin{CodeNoLabel}
\prompt{\$ CC=icc mkcmake}
checking for compiler type... \h{icc}
checking for function strlcpy... \h{no}
checking for func strlcpy ( string.h )... \h{no}
checking for program icc... /opt/intel/cc/10.1.008/bin/icc
icc -c main.c
icc -c strlcpy.c
icc -o strlcpy_test main.o \h{strlcpy.o}
\prompt{\$ echo \_mkc\_*}
_mkc_compiler_type.err _mkc_compiler_type.res 
_mkc_func3_strlcpy_string_h.c 
_mkc_func3_strlcpy_string_h.err 
_mkc_func3_strlcpy_string_h.res 
_mkc_funclib_strlcpy.c _mkc_funclib_strlcpy.err 
_mkc_funclib_strlcpy.res _mkc_prog_cc.err _mkc_prog_cc.res
\prompt{\$}
\end{CodeNoLabel}
  \end{block}
\end{frame}

%%%%%%%%%%%%%%%%%%%%%%%%%%%%%%%%%%%%%%%%%%%%%%%%%%%%%%%%%%%%%%%%%%%%%%
\begin{frame}[fragile]
  \frametitle{Example 2: Application using non-standard strlcpy(3)}

  \begin{block}{How it works on NetBSD}
\begin{CodeNoLabel}
\prompt{\$ mkcmake}
checking for compiler type... gcc
checking for function strlcpy... \h{yes}
checking for func strlcpy ( string.h )... \h{yes}
checking for program cc... /usr/bin/cc
cc -D\h{HAVE\_FUNC3\_STRLCPY\_STRING\_H}=1    -c main.c
cc -o strlcpy_test main.o
\prompt{\$} 
\end{CodeNoLabel}
  \end{block}
\end{frame}

%%%%%%%%%%%%%%%%%%%%%%%%%%%%%%%%%%%%%%%%%%%%%%%%%%%%%%%%%%%%%%%%%%%%%%
\begin{frame}[fragile]
  \frametitle{Example 3: Application using plugins}

  \begin{block}{Source code}
  \begin{Code}{Makefile}
PROG =   myapp

\h{MKC\_CHECK\_FUNCLIBS} =     dlopen:dl

.include <\h{mkc.configure.mk}>

.if $\{\h{HAVE\_FUNCLIB.dlopen}:U0\} || \ {}
    $\{\h{HAVE\_FUNCLIB.dlopen.dl}:U0\}
CFLAGS +=	-DPLUGINS_ENABLED=1
.endif

.include <mkc.prog.mk>
  \end{Code}
  \end{block}
\end{frame}

%%%%%%%%%%%%%%%%%%%%%%%%%%%%%%%%%%%%%%%%%%%%%%%%%%%%%%%%%%%%%%%%%%%%%%
\begin{frame}[fragile]
  \frametitle{Example 3: Application using plugins}

  \begin{block}{How it works on QNX}
\begin{CodeNoLabel}
\prompt{\$ mkcmake}
checking for compiler type... gcc
checking for function dlopen ( \h{-ldl} )... \h{yes}
checking for function dlopen... \h{no}
checking for program gcc...
      /usr/qnx650/host/qnx6/x86/usr/bin/gcc
gcc \h{-DPLUGINS\_ENABLED=1}    -c myapp.c
gcc -o myapp myapp.o \h{-ldl}
\prompt{\$}
\end{CodeNoLabel}
  \end{block}
\end{frame}

%%%%%%%%%%%%%%%%%%%%%%%%%%%%%%%%%%%%%%%%%%%%%%%%%%%%%%%%%%%%%%%%%%%%%%
\begin{frame}[fragile]
  \frametitle{Example 3: Application using plugins}

  \begin{block}{How it works on OpenBSD}
\begin{CodeNoLabel}
\prompt{\$ mkcmake}
checking for compiler type... gcc
checking for function dlopen ( \h{-ldl} )... \h{no}
checking for function dlopen... \h{yes}
checking for program cc... /usr/bin/cc
cc \h{-DPLUGINS\_ENABLED=1}    -c myapp.c
cc -o myapp myapp.o
\prompt{\$}
\end{CodeNoLabel}
  \end{block}
\end{frame}

%%%%%%%%%%%%%%%%%%%%%%%%%%%%%%%%%%%%%%%%%%%%%%%%%%%%%%%%%%%%%%%%%%%%%%
\begin{frame}[fragile]
  \frametitle{Example 4: Support for shared libraries and C++}

  \begin{block}{Source code}
  \begin{Code}{Makefile}
LIB =                   foobar
SRCS =                  foo.\h{cc} bar.\h{cc} baz.\h{cc}

MKPICLIB ?=             no
MKSTATICLIB ?=          no

\h{SHLIB\_MAJOR} =           1
\h{SHLIB\_MINOR} =           0

.include <\h{mkc.lib.mk}>
  \end{Code}
  \end{block}
\end{frame}

%%%%%%%%%%%%%%%%%%%%%%%%%%%%%%%%%%%%%%%%%%%%%%%%%%%%%%%%%%%%%%%%%%%%%%
\begin{frame}[fragile]
  \frametitle{Example 4: Support for shared libraries}

  \begin{block}{How it works on Solaris with SunStudio compiler}
\begin{CodeNoLabel}
\prompt{\$ mkcmake}
checking for compiler type... \h{sunpro}
checking for program CC... /opt/SUNWspro/bin
CC -c \h{-KPIC} foo.cc -o foo.os
CC -c \h{-KPIC} bar.cc -o bar.os
CC -c \h{-KPIC} baz.cc -o baz.os
building shared foobar library (version \h{1.0})
CC \h{-G} \h{-h libfoobar.so.1}
   -o libfoobar.so.1.0  foo.os bar.os baz.os
ln -sf libfoobar.so.1.0 libfoobar.so
ln -sf libfoobar.so.1.0 libfoobar.so.1
\prompt{\$}
\end{CodeNoLabel}
  \end{block}
\end{frame}

%%%%%%%%%%%%%%%%%%%%%%%%%%%%%%%%%%%%%%%%%%%%%%%%%%%%%%%%%%%%%%%%%%%%%%
\begin{frame}[fragile]
  \frametitle{Example 4: Support for shared libraries}

  \begin{block}{How it works on Darwin}
\begin{CodeNoLabel}
\prompt{\$ mkcmake}
checking for compiler type... gcc
checking for program c++... /usr/bin/c++
c++    -c \h{-fPIC -DPIC} foo.cc -o foo.os
c++    -c \h{-fPIC -DPIC} bar.cc -o bar.os
c++    -c \h{-fPIC -DPIC} baz.cc -o baz.os
building shared foobar library (version 1.0)
c++ \h{-dynamiclib -install\_name}
   /usr/local/lib/libfoobar.1.0.\h{dylib}
   \h{-current\_version 2.0 -compatibility\_version 2} 
   -o libfoobar.1.0.dylib  foo.os bar.os baz.os
ln -sf libfoobar.1.0.dylib libfoobar.dylib
ln -sf libfoobar.1.0.dylib libfoobar.1.dylib
\prompt{\$}
\end{CodeNoLabel}
  \end{block}
\end{frame}

%%%%%%%%%%%%%%%%%%%%%%%%%%%%%%%%%%%%%%%%%%%%%%%%%%%%%%%%%%%%%%%%%%%%%%
\begin{frame}[fragile]
  \frametitle{Example 4: Support for shared libraries (Exported symbols)}

  \begin{block}{}
\begin{CodeNoLabel}
\prompt{\$ cat Makefile}
LIB            = foo
INCS           = foo.h
\h{EXPORT\_SYMBOLS} = foo.sym
SHLIB_MAJOR    = 1
MKSTATICLIB    = no
.include <mkc.lib.mk>
\prompt{\$ mkcmake}
awk 'BEGIN \{print "\{ global:"\}  \{print \$0 ";"\}
     END \{print "local: *; \};"\}' foo.sym
     > foo.sym.tmp1 && mv foo.sym.tmp1 foo.sym.tmp
cc    -I.  -c -fPIC -DPIC foo.c -o foo.os
building shared foo library (version 1)
ld -shared -soname libfoo.so.1
   \h{--version-script foo.sym.tmp} -o libfoo.so.1  foo.os
\prompt{\$}
\end{CodeNoLabel}
  \end{block}
\end{frame}

%%%%%%%%%%%%%%%%%%%%%%%%%%%%%%%%%%%%%%%%%%%%%%%%%%%%%%%%%%%%%%%%%%%%%%
\begin{frame}[fragile]
  \frametitle{Example 5: Big project consisting of several subprojects}

  \begin{block}{Dependency graph for all subprojects}
This project consists of several subprojects: dict, dictd, dictfmt,
dictzip, libdz, libmaa and libcommon. libcommon contains common code
for executables and should not be installed.
    \begin{figure}
      \includegraphics[width=\textwidth, height=0.50\textheight, keepaspectratio=false]{dep_graph.eps}
    \end{figure}
  \end{block}
\end{frame}

%%%%%%%%%%%%%%%%%%%%%%%%%%%%%%%%%%%%%%%%%%%%%%%%%%%%%%%%%%%%%%%%%%%%%%
\begin{frame}[fragile]
  \frametitle{Example 5: Big project consisting of several subprojects}

  \begin{block}{Files and directories}
    \begin{CodeNoLabel}
\prompt{\$ ls -l}
total 4
drwxr-xr-x 2 cheusov users   1 Jan 26 12:01 dict
drwxr-xr-x 2 cheusov users   1 Jan 26 12:01 dictd
drwxr-xr-x 2 cheusov users   1 Jan 26 12:01 dictfmt
drwxr-xr-x 2 cheusov users   1 Jan 26 12:01 dictzip
drwxr-xr-x 2 cheusov users   1 Jan 26 12:03 doc
drwxr-xr-x 2 cheusov users   1 Jan 26 12:01 libcommon
drwxr-xr-x 2 cheusov users   1 Jan 26 12:01 libdz
drwxr-xr-x 2 cheusov users   1 Jan 26 12:01 libmaa
-rw-r--r-- 1 cheusov users 306 Jan 26 12:03 \h{Makefile}
\prompt{\$}
    \end{CodeNoLabel}
  \end{block}
\end{frame}

%%%%%%%%%%%%%%%%%%%%%%%%%%%%%%%%%%%%%%%%%%%%%%%%%%%%%%%%%%%%%%%%%%%%%%
\begin{frame}[fragile]
  \frametitle{Example 5: Big project consisting of several subprojects}

  \begin{block}{Source code}
  \begin{Code}{Makefile}
SUBPRJ =   libcommon:dict   # dict depends on libcommon
SUBPRJ +=  libcommon:dictd
SUBPRJ +=  libcommon:dictzip
SUBPRJ +=  libcommon:dictfmt
SUBPRJ +=  libmaa:dict
SUBPRJ +=  libmaa:dictd
SUBPRJ +=  libmaa:dictfmt
SUBPRJ +=  libmaa:dictzip
SUBPRJ +=  libdz:dictzip
SUBPRJ +=  doc

.include <\h{mkc.subprj.mk}>
  \end{Code}
  \end{block}
\end{frame}

%%%%%%%%%%%%%%%%%%%%%%%%%%%%%%%%%%%%%%%%%%%%%%%%%%%%%%%%%%%%%%%%%%%%%%
\begin{frame}[fragile]
  \frametitle{Example 5: Big project consisting of several subprojects}

  \begin{block}{Source code}
  \begin{Code}{libcommon/Makefile}
# Internal static library that implements functions
# common for dict, dictd, dictfmt and dictzip applications

LIB =            common
SRCS =           str.c iswalnum.c # and others

\h{MKINSTALL} =   no # Do not install internal library!

.include <mkc.lib.mk>
  \end{Code}
  \begin{Code}{libcommon/linkme.mk}
PATH.common :=      \$\{.PARSEDIR\}

CPPFLAGS +=      -I\$\{PATH.common\}\h{/include}
DPLIBDIRS +=     \$\{PATH.common\}
  \end{Code}
  \end{block}
\end{frame}

%%%%%%%%%%%%%%%%%%%%%%%%%%%%%%%%%%%%%%%%%%%%%%%%%%%%%%%%%%%%%%%%%%%%%%
\begin{frame}[fragile]
  \frametitle{Example 5: Big project consisting of several subprojects}

  \begin{block}{Source code}
  \begin{Code}{libmaa/Makefile}
LIB =       maa
SRCS =      set.c prime.c log.c # etc.

\h{INCS} =      maa.h

SHLIB_MAJOR =    1
SHLIB_MINOR =    2
SHLIB_TEENY =    0

# list of exported symbols
\h{EXPORT\_SYMBOLS} = maa.sym

.include <mkc.lib.mk>
  \end{Code}
  \begin{Code}{libmaa/linkme.mk}
PATH.maa :=      \$\{.PARSEDIR\}
CPPFLAGS +=      -I\$\{PATH.maa\}
DPLIBDIRS +=     \$\{PATH.maa\}
  \end{Code}
  \end{block}
\end{frame}

%%%%%%%%%%%%%%%%%%%%%%%%%%%%%%%%%%%%%%%%%%%%%%%%%%%%%%%%%%%%%%%%%%%%%%
\begin{frame}[fragile]
  \frametitle{Example 5: Big project consisting of several subprojects}

  \begin{block}{Source code}
  \begin{Code}{libmaa/maa.sym}
hsh_create
hsh_destroy
hsh_insert
hsh_delete
hsh_retrieve
...
lst_create
lst_destroy
lst_insert
...
set_create
set_destroy
set_add
set_union
...
  \end{Code}
  \end{block}
\end{frame}

%%%%%%%%%%%%%%%%%%%%%%%%%%%%%%%%%%%%%%%%%%%%%%%%%%%%%%%%%%%%%%%%%%%%%%
\begin{frame}[fragile]
  \frametitle{Example 5: Big project consisting of several subprojects}

  \begin{block}{Source code}
  \begin{Code}{libdz/Makefile}
LIB =            dz
SRCS =           dz.c

INCS =           dz.h

\h{MKC\_REQUIRE\_HEADERS} =    zlib.h
\h{MKC\_REQUIRE\_FUNCLIBS} =   deflate:z
EXPORT_SYMBOLS =         dz.sym
SHLIB_MAJOR =            1
SHLIB_MINOR =            0
LDADD =                  -lz

.include <mkc.lib.mk>
  \end{Code}
  \begin{Code}{libdz/linkme.mk}
PATH.dz :=       \$\{.PARSEDIR\}
CPPFLAGS +=      -I\$\{PATH.dz\}
DPLIBDIRS +=     \$\{PATH.dz\}
  \end{Code}
  \end{block}
\end{frame}

%%%%%%%%%%%%%%%%%%%%%%%%%%%%%%%%%%%%%%%%%%%%%%%%%%%%%%%%%%%%%%%%%%%%%%
\begin{frame}[fragile]
  \frametitle{Example 5: Big project consisting of several subprojects}

  \begin{block}{Source code}
  \begin{Code}{dictzip/Makefile}
PROG =   dictzip
MAN =    dictzip.1

\h{.include} "../libcommon/linkme.mk"
\h{.include} "../libdz/linkme.mk"
\h{.include} "../libmaa/linkme.mk"

DPLIBS +=        -lcommon -ldz -lmaa

.include <mkc.prog.mk>
  \end{Code}
  \end{block}
\end{frame}

%%%%%%%%%%%%%%%%%%%%%%%%%%%%%%%%%%%%%%%%%%%%%%%%%%%%%%%%%%%%%%%%%%%%%%
\begin{frame}[fragile]
  \frametitle{Example 5: Big project consisting of several subprojects}

  \begin{block}{How it fails ;-)}
  \begin{CodeNoLabel}
\prompt{\$ mkcmake errorcheck-dictzip}
==================================================
errorcheck ===> \h{libcommon}
...
errorcheck ===> \h{libmaa}
...
==================================================
errorcheck ===> \h{libdz}
checking for header zlib.h... \h{no}
checking for function deflate ( -lz )... \h{no}
checking for function deflate... \h{no}
\h{ERROR: cannot find header zlib.h}
\h{ERROR: cannot find function deflate:z}
...
\prompt{\$ echo \$?}
1
\prompt{\$}
  \end{CodeNoLabel}
  \end{block}
\end{frame}

%%%%%%%%%%%%%%%%%%%%%%%%%%%%%%%%%%%%%%%%%%%%%%%%%%%%%%%%%%%%%%%%%%%%%%
\begin{frame}[fragile]
  \frametitle{Example 5: Big project consisting of several subprojects}

  \begin{block}{How it works}
  \begin{CodeNoLabel}
\prompt{\$ mkcmake dictzip}
...
==================================================
all ===> \h{libdz}
...
checking for header zlib.h... \h{yes}
checking for function deflate ( -lz )... \h{yes}
...
==================================================
all ===> \h{dictzip}
...
cc \h{-I../libcommon -I../libdz -I../libmaa} -c dictzip.c
cc \h{-L/tmp/hello\_dictd/libcommon -L/tmp/hello\_dictd/libdz}
   \h{-L/tmp/hello\_dictd/libmaa}  -o dictzip
   dictzip.o -lcommon -lmaa -ldz
\prompt{\$}
  \end{CodeNoLabel}
  \end{block}
\end{frame}

%%%%%%%%%%%%%%%%%%%%%%%%%%%%%%%%%%%%%%%%%%%%%%%%%%%%%%%%%%%%%%%%%%%%%%
\begin{frame}[fragile]
  \frametitle{Example 6: Support for Lua programming language}

  \begin{block}{Source code}
  \begin{Code}{Makefile}
SCRIPTS =       foobar  # scripts written in Lua
LUA\_LMODULES =  foo bar # modules written in Lua
LUA\_CMODULE =   baz     # Lua module written in C

.include <mkc.lib.mk>
  \end{Code}
  \end{block}
\end{frame}

%%%%%%%%%%%%%%%%%%%%%%%%%%%%%%%%%%%%%%%%%%%%%%%%%%%%%%%%%%%%%%%%%%%%%%
\begin{frame}[fragile]
  \frametitle{Example 6: Support for Lua programming language}

  \begin{block}{How it works}
  \begin{CodeNoLabel}
\prompt{\$ mkcmake errorcheck}
checking for program pkg-config...
   /usr/pkg/bin/pkg-config
checking for [pkg-config] lua... 1 (yes)
checking for [pkg-config] lua --cflags...
   -I/usr/pkg/include
checking for [pkg-config] lua --libs...
   -Wl,-R/usr/pkg/lib -L/usr/pkg/lib -llua -lm
checking for [pkg-config] lua --variable=INSTALL_LMOD...
   /usr/pkg/share/lua/5.1
checking for [pkg-config] lua --variable=INSTALL_CMOD...
   /usr/pkg/lib/lua/5.1
checking for compiler type... gcc
checking for header lua.h... yes
checking for program cc... /usr/bin/cc
\prompt{\$}
  \end{CodeNoLabel}
  \end{block}
\end{frame}

%%%%%%%%%%%%%%%%%%%%%%%%%%%%%%%%%%%%%%%%%%%%%%%%%%%%%%%%%%%%%%%%%%%%%%
\begin{frame}[fragile]
  \frametitle{Example 6: Support for Lua programming language}

  \begin{block}{How it works}
  \begin{CodeNoLabel}
\prompt{\$ export PREFIX=/usr/pkg}
\prompt{\$ mkcmake all}
cc -DHAVE_HEADER_LUA_H=1 -I/usr/pkg/include
   -c -fPIC -DPIC baz.c -o baz.os
building shared baz library (version 1.0)
cc -shared -Wl,-soname -Wl,libbaz.so.1 -o baz.so  baz.os
   -Wl,-R/usr/pkg/lib -L/usr/pkg/lib -llua -lm
\prompt{\$}
  \end{CodeNoLabel}
  \end{block}
\end{frame}

%%%%%%%%%%%%%%%%%%%%%%%%%%%%%%%%%%%%%%%%%%%%%%%%%%%%%%%%%%%%%%%%%%%%%%
\begin{frame}[fragile]
  \frametitle{Example 6: Support for Lua programming language}

  \begin{block}{How it works}
  \begin{CodeNoLabel}
\prompt{\$ mkcmake install DESTDIR=/tmp/fakeroot}
    ...
\prompt{\$ find /tmp/fakeroot -type f}
/tmp/fakeroot/usr/pkg/bin/foobar
/tmp/fakeroot/usr/pkg/lib/lua/5.1/baz.so
/tmp/fakeroot/usr/pkg/share/lua/5.1/foo.lua
/tmp/fakeroot/usr/pkg/share/lua/5.1/bar.lua
\prompt{\$}
  \end{CodeNoLabel}
  \end{block}
\end{frame}

%%%%%%%%%%%%%%%%%%%%%%%%%%%%%%%%%%%%%%%%%%%%%%%%%%%%%%%%%%%%%%%%%%%%%%
\begin{frame}[fragile]
  \frametitle{Example 7: Custom tests and sizeof of system types}

\begin{block}{Source code}
\begin{Code}{Makefile}
MKC\_CUSTOM\_DIR      = \h{\$\{.CURDIR\}/checks}

# m4 supports -P flag (GNU, NetBSD)
M4                 ?= m4 # overridable (\h{gm4})
MKC\_REQUIRE\_CUSTOM += m4P
MKC\_CUSTOM\_FN.m4P   = \h{m4P.sh}
.export: M4

# \_\_attribute ((\{con,de\}structor))
MKC\_REQUIRE\_CUSTOM += \h{constructor destructor}

# sizeof
MKC\_CHECK\_SIZEOF    = char short int long void* long-long

LIB                 = mylib
CFLAGS             += -DM4\_CMD='"\$\{M4\}"'
.include <mkc.lib.mk>
\end{Code}
\end{block}
\end{frame}

%%%%%%%%%%%%%%%%%%%%%%%%%%%%%%%%%%%%%%%%%%%%%%%%%%%%%%%%%%%%%%%%%%%%%%
\begin{frame}[fragile]
  \frametitle{Example 7: Custom tests and sizeof of system types}

\begin{block}{Source code}
\begin{Code}{Files in checks/ subdirectory}
\prompt{\$ ls checks/}
constructor.c
destructor.c
m4P.sh
\prompt{\$}
\end{Code}
\end{block}
\end{frame}

%%%%%%%%%%%%%%%%%%%%%%%%%%%%%%%%%%%%%%%%%%%%%%%%%%%%%%%%%%%%%%%%%%%%%%
\begin{frame}[fragile]
  \frametitle{Example 7: Custom tests and sizeof of system types}

\begin{block}{Source code}
\begin{Code}{checks/m4P.sh}
#!/bin/sh

input ()\{
    cat <<'EOF'
m4\_define(fruit, apple)
fruit
EOF
\}

M4=\$\{M4-m4\}

if input | \$\{M4\} -P | grep \^{}apple > /dev/null; then
    echo 1
else
    echo 0
fi
\end{Code}
\end{block}
\end{frame}

%%%%%%%%%%%%%%%%%%%%%%%%%%%%%%%%%%%%%%%%%%%%%%%%%%%%%%%%%%%%%%%%%%%%%%
\begin{frame}[fragile]
  \frametitle{Example 7: Custom tests and sizeof of system types}
\begin{block}{Source code}
\begin{Code}{checks/constructor.c}
void __attribute ((constructor))
        dummy (void)
\{
\}
\end{Code}
\begin{Code}{checks/destructor.c}
void __attribute ((destructor))
        dummy (void)
\{
\}
\end{Code}
\end{block}
\end{frame}

%%%%%%%%%%%%%%%%%%%%%%%%%%%%%%%%%%%%%%%%%%%%%%%%%%%%%%%%%%%%%%%%%%%%%%
\begin{frame}[fragile]
  \frametitle{Example 7: Custom tests and sizeof of system types}

  \begin{block}{How it works on FreeBSD}
  \begin{CodeNoLabel}
\prompt{\$ mkcmake}
checking for compiler type... gcc
checking for sizeof char... \h{1}
checking for sizeof short... \h{2}
checking for sizeof int... \h{4}
checking for sizeof long... \h{4}
checking for sizeof void*... \h{4}
checking for sizeof long long... \h{8}
checking for custom test \h{m4P... 0 (no)}
checking for custom test \h{constructor}... \h{1 (yes)}
checking for custom test \h{destructor}... \h{1 (yes)}
checking for program cc... /usr/bin/cc
\h{ERROR: custom test m4P failed}
*** Error code 1
...
\prompt{\$}
  \end{CodeNoLabel}
  \end{block}
\end{frame}

%%%%%%%%%%%%%%%%%%%%%%%%%%%%%%%%%%%%%%%%%%%%%%%%%%%%%%%%%%%%%%%%%%%%%%
\begin{frame}[fragile]
  \frametitle{Example 7: Custom tests and sizeof of system types}

  \begin{block}{How it works on FreeBSD with GNU m4}
  \begin{CodeNoLabel}
\prompt{\$ M4=gm4 mkcmake}
checking for compiler type... gcc
checking for sizeof char... 1
checking for sizeof short... 2
checking for sizeof int... 4
checking for sizeof long... 4
checking for sizeof void*... 4
checking for sizeof long long... 8
checking for custom test \h{m4P... 1 (yes)}
checking for custom test constructor... 1 (yes)
checking for custom test destructor... 1 (yes)
checking for program cc... /usr/bin/cc
...
\prompt{\$}
  \end{CodeNoLabel}
  \end{block}
\end{frame}

%%%%%%%%%%%%%%%%%%%%%%%%%%%%%%%%%%%%%%%%%%%%%%%%%%%%%%%%%%%%%%%%%%%%%%
\begin{frame}[fragile]
  \frametitle{Example 8: Portable version of AWK from NetBSD}

\begin{block}{http://mova.org/\~{}cheusov/pub/mk-configure/nbawk/}
\begin{Code}{Makefile (part 1)}
PROG =	awk
SRCS =	awkgram.y b.c lex.c lib.c main.c parse.c
        proctab.c run.c tran.c
YHEADER =	yes

MKC\_COMMON\_DEFINES.Linux =	-D\_GNU\_SOURCE
MKC\_COMMON\_HEADERS =		ctype.h stdio.h string.h
MKC\_CHECK\_FUNCS1 =		\_\_fpurge:stdio\_ext.h fpurge isblank
MKC\_CHECK\_FUNCS3 =		strlcat
MKC\_SOURCE\_FUNCLIBS =		fpurge strlcat

WARNS=       4
WARNERR=     no # do not treat warnings as errors

MKC\_REQD=   0.19.0 # mk-configure>=0.19.0 is required
... # to be continued on the next slide
\end{Code}
\end{block}
\end{frame}

%%%%%%%%%%%%%%%%%%%%%%%%%%%%%%%%%%%%%%%%%%%%%%%%%%%%%%%%%%%%%%%%%%%%%%
\begin{frame}[fragile]
  \frametitle{Example 8: Portable version of AWK from NetBSD}

\begin{block}{http://mova.org/\~{}cheusov/pub/mk-configure/nbawk/}
\begin{Code}{Makefile (part 2)}
... # beginning is on the previous slide
\h{.include} <mkc.configure.mk>

\h{.if} \$\{HAVE\_FUNC1.isblank:U0\}
CPPFLAGS +=	-DHAS\_ISBLANK
\h{.endif}

\h{.if} !\$\{HAVE\_FUNC1.fpurge:U1\} &&
   !\$\{HAVE_FUNC1.\_\_fpurge.stdio\_ext\_h:U1\}
MKC\_ERR\_MSG+=	"fpurge(3) cannot be found"
\h{.endif}

CPPFLAGS +=     -I.
LDADD +=        -lm

\h{.include} <mkc.prog.mk>
\end{Code}
\end{block}
\end{frame}

%%%%%%%%%%%%%%%%%%%%%%%%%%%%%%%%%%%%%%%%%%%%%%%%%%%%%%%%%%%%%%%%%%%%%%
\begin{frame}[fragile]
  \frametitle{Example 8: Portable version of AWK from NetBSD}

\begin{block}{run.c}
\begin{CodeNoLabelSmallest}
--- nbawk-20100903/run.c.orig
+++ nbawk-20100903/run.c
@@ -40,6 +40,14 @@
 #include "awk.h"
 #include "awkgram.h"
 
+#ifndef HAVE_FUNC1_FPURGE
+int fpurge (FILE *stream);
+#endif
+
+#ifndef HAVE\_FUNC3\_STRLCAT
+size\_t strlcat(char *dst, const char *src, size_t size);
+#endif
+
 #define tempfree(x)    if (istemp(x)) tfree(x); else
 
 void stdinit(void);
\end{CodeNoLabelSmallest}
\end{block}
\end{frame}

%%%%%%%%%%%%%%%%%%%%%%%%%%%%%%%%%%%%%%%%%%%%%%%%%%%%%%%%%%%%%%%%%%%%%%
\begin{frame}[fragile]
  \frametitle{Example 8: Portable version of AWK from NetBSD}

\begin{block}{fpurge.c}
\begin{CodeNoLabelSmallest}
#include <stdio.h>

#if HAVE\_FUNC1\_\_\_FPURGE\_STDIO\_EXT\_H
#include <stdio\_ext.h>
#endif

int fpurge(FILE *stream);

int fpurge(FILE *stream)
\{
#if HAVE\_FUNC1\_\_\_FPURGE\_STDIO\_EXT\_H
  \_\_fpurge (stream);
  return 0;
#else
#error "cannot find fpurge(3), sorry"
#endif
\}
\end{CodeNoLabelSmallest}
\end{block}
\end{frame}

%%%%%%%%%%%%%%%%%%%%%%%%%%%%%%%%%%%%%%%%%%%%%%%%%%%%%%%%%%%%%%%%%%%%%%
\begin{frame}[fragile]
  \frametitle{Example 8: Portable version of AWK from NetBSD}

\begin{block}{strlcpy.c}
\begin{CodeNoLabel}
If you need this code, you know where to get it from! ;-)
\end{CodeNoLabel}
\end{block}
\end{frame}

%%%%%%%%%%%%%%%%%%%%%%%%%%%%%%%%%%%%%%%%%%%%%%%%%%%%%%%%%%%%%%%%%%%%%%
\begin{frame}[fragile]
  \frametitle{Example 8: Portable version of AWK from NetBSD}

%  \begin{block}
\begin{block}{How it works on Linux}
\begin{CodeNoLabel}
\prompt{\$ mkcmake}
checking for compiler type... gcc
checking for function fpurge... \h{no}
checking for function strlcat... \h{no}
checking for func __fpurge ( stdio_ext.h )... \h{yes}
checking for func fpurge... \h{no}
checking for func isblank... \h{yes}
checking for func strlcat... \h{no}
checking for program yacc... \h{/usr/bin/yacc}
...
cc \h{-Wall -Wstrict-prototypes ...} 
    -I. \h{-D\_GNU\_SOURCE} -c awkgram.c
...
cc -o awk awkgram.o ... \h{fpurge.o strlcat.o} -lm
\prompt{\$ ./awk}
usage: ./awk [-F fs] [-v var=value] [-f progfile 
    | 'prog'] [file ...]
\prompt{\$}
\end{CodeNoLabel}
\end{block}
Supported targets: all, clean, cleandir (distclean), install,
uninstall, installdirs, depend etc.
%  \end{block}
\end{frame}

%%%%%%%%%%%%%%%%%%%%%%%%%%%%%%%%%%%%%%%%%%%%%%%%%%%%%%%%%%%%%%%%%%%%%%
\begin{frame}[fragile]
  \frametitle{Example 9: Cross-compilation}

\begin{block}{How it works}
\begin{CodeNoLabel}
\prompt{\$ export SYSROOT=/tmp/destdir.sparc64}
\prompt{\$ export TOOLCHAIN\_PREFIX=sparc64--netbsd-}
\prompt{\$ export TOOLCHAIN\_DIR=/tmp/tooldir.sparc64/bin}
\prompt{\$ uname -srm}
NetBSD 5.99.56 amd64
\prompt{\$ mkcmake}
checking for compiler type... gcc
/tmp/tooldir.sparc64/bin/sparc64--netbsd-gcc
   --sysroot=/tmp/destdir.sparc64 -c hello.c -o hello.o
/tmp/tooldir.sparc64/bin/sparc64--netbsd-gcc
   --sysroot=/tmp/destdir.sparc64 -o hello hello.o
\prompt{\$ file hello}
hello: ELF 64-bit MSB executable, SPARC V9, relaxed
   memory ordering, (SYSV), dynamically linked (uses
   shared libs), for NetBSD 5.99.56, not stripped
\prompt{\$ }
\end{CodeNoLabel}
\end{block}
\end{frame}

%%%%%%%%%%%%%%%%%%%%%%%%%%%%%%%%%%%%%%%%%%%%%%%%%%%%%%%%%%%%%%%%%%%%%%
\begin{frame}[fragile]
  \frametitle{Features}
  \begin{block}{}
  \begin{enumerate}
  \item Automatic detection of OS features
    (\ModuleName{mkc.configure.mk})
    \begin{itemize}
    \item \h{header presence} (MKC\_\{CHECK,REQUIRE\}\_HEADERS)
    \item \h{function declaration} (MKC\_\{CHECK,REQUIRE\}\_FUNCS[n])
    \item \h{type declaration} (MKC\_\{CHECK,REQUIRE\}\_TYPES)
    \item \h{structure member} (MKC\_\{CHECK,REQUIRE\}\_MEMBERS)
    \item \h{variable declaration} (MKC\_\{CHECK,REQUIRE\}\_VARS)
    \item \h{define declaration} (MKC\_\{CHECK,REQUIRE\}\_DEFINES)
    \item \h{type size} (MKC\_CHECK\_SIZEOF)
    \item \h{function} implementation \h{in the library}
      (MKC\_\{CHECK,REQUIRE\}\_FUNCLIBS and MKC\_SOURCE\_FUNCLIBS)
    \item \h{checks for programs} (MKC\_\{CHECK,REQUIRE\}\_PROGS)
    \item \h{user's custom checks} (MKC\_\{CHECK,REQUIRE\}\_CUSTOM)
    \item \h{built-in checks} (MKC\_CHECK\_BUILTINS), e.g. endianess,
      prog\_flex, prog\_bison, prog\_gawk or prog\_gm4)
    \end{itemize}
  \end{enumerate}
  \end{block}
\end{frame}

%%%%%%%%%%%%%%%%%%%%%%%%%%%%%%%%%%%%%%%%%%%%%%%%%%%%%%%%%%%%%%%%%%%%%%
\begin{frame}[fragile,t]
  \frametitle{Features}
  \begin{block}{}
  \begin{enumerate}
  \setcounter{enumi}{1}
  \item Building, installing, uninstalling, cleaning
    etc. Supported targets: \h{all}, \h{install},
    \h{uninstall},
    \h{clean}, \h{cleandir} (\h{distclean}), \h{installdirs}, \h{depend}
    and others.
  \item Building \h{standalone programs} (\ModuleName{mkc.prog.mk}),
    \h{static, shared and dynamically loaded libraries}
    (\ModuleName{mkc.lib.mk}) using \h{C},
    \h{C++}, \h{Objective C}, \h{Pascal} and \h{Fortran} compilers.
    Shared libraries
    support is provided for numerous OSes: \h{NetBSD}, \h{FreeBSD},
    \h{OpenBSD},
    \h{DragonFlyBSD}, \h{MirOS BSD}, \h{Linux}, \h{Solaris}, \h{Darwin}
    (MacOS-X), \h{Interix}, \h{Tru64},
    \h{QNX}, \h{HP-UX}, \h{Cygwin} (no support for shared libraries
    and DLLs yet) and
    compilers: \h{GCC}, \h{Intel C/C++} compilers, \h{Portable C compiler} AKA
    \h{pcc}, \h{DEC C/C++ compiler}, \h{HP C/C++ compiler},
    \h{Oracle SunStudio} and others.
  \end{enumerate}
  \end{block}
\end{frame}

%%%%%%%%%%%%%%%%%%%%%%%%%%%%%%%%%%%%%%%%%%%%%%%%%%%%%%%%%%%%%%%%%%%%%%

\begin{frame}[fragile,t]
  \frametitle{Features}
  \begin{block}{}
  \begin{enumerate}
  \setcounter{enumi}{3}
  \item Handling \h{man} pages, \h{info} manuals and \h{POD} documents.
  \item Handling \h{scripts} as well as \h{plain text files},
    i.e. installing, uninstalling and handling \h{.in files}
    (replacing, for example, \h{@bindir@}, \h{@sysconfdir@}, \h{@version@}
    fragments with real values).
  \item \h{Cross-compilation}. mk-configure itself doesn't run
    target system
    executables, so you can freely use cross-tools (compiler, linker
    etc.).  You can also override/set any variable initialized by mk-configure.
  \item Support for \h{pkg-config}.
  \item Support for \h{Lua} programming language.
  \item Support for \h{yacc} and \h{lex}.
  \end{enumerate}
  \end{block}
\end{frame}

%%%%%%%%%%%%%%%%%%%%%%%%%%%%%%%%%%%%%%%%%%%%%%%%%%%%%%%%%%%%%%%%%%%%%%

\begin{frame}[fragile,t]
  \frametitle{Features}
  \begin{block}{}
  \begin{enumerate}
    \setcounter{enumi}{10}
    \item Support for projects with multiple subprojects
    (\ModuleName{mkc.subprj.mk} and \ModuleName{mkc.subdir.mk}).
    \item Special targets bin\_tar, bin\_targz, bin\_tarbz2, bin\_zip,
      bin\_deb creates .tar, .tar.gz, .tar.bz2, .zip archives and .deb
      package (on Debian Linux).
    \item Parts of mk-configure functionality is
      accesible via individual programs, e.g.  \ProgName{mkc\_install},
      \ProgName{mkc\_check\_compiler},
      \ProgName{mkc\_check\_header}, \ProgName{mkc\_check\_funclib},
      \ProgName{mkc\_check\_decl},
      \ProgName{mkc\_check\_prog}, \ProgName{mkc\_check\_sizeof} and
      \ProgName{mkc\_check\_custom}.
  \end{enumerate}
  \end{block}
\end{frame}

%%%%%%%%%%%%%%%%%%%%%%%%%%%%%%%%%%%%%%%%%%%%%%%%%%%%%%%%%%%%%%%%%%%%%%

\begin{frame}[fragile,t]
  \frametitle{MK-CONFIGURE in real world}
  \begin{block}{Packagers are welcome! ;-)}
    \small
    NetBSD make (bmake) is packaged in the following OSes:
    \begin{itemize}
    \item FreeBSD, pkgsrc (NetBSD, DragonFlyBSD...) (\h{devel/bmake})
    \item Gentoo Linux, Fedora Linux, AltLinux
    \item Debian/Ubuntu\\
      deb http://mova.org/\~{}cheusov/pub/debian lenny main\\
      deb-src http://mova.org/\~{}cheusov/pub/debian lenny main
    \end{itemize}
    mk-configure is packaged in the following OSes
    \begin{itemize}
    \item FreeBSD, pkgsrc (NetBSD, DragonFlyBSD...) (\h{devel/mk-configure})
    \item Debian/Ubuntu\\
      deb http://mova.org/\~{}cheusov/pub/debian lenny main\\
      deb-src http://mova.org/\~{}cheusov/pub/debian lenny main
    \end{itemize}
  \end{block}
\end{frame}

%%%%%%%%%%%%%%%%%%%%%%%%%%%%%%%%%%%%%%%%%%%%%%%%%%%%%%%%%%%%%%%%%%%%%%

\begin{frame}[fragile,t]
  \frametitle{MK-CONFIGURE in real world}
  \begin{block}{Real life samples of use}
  \begin{itemize}
  \item Lightweight modular malloc Debugger.\\
    \URL{http://sourceforge.net/projects/lmdbg/}
    \URL{http://pkgsrc.se/wip/lmdbg/}\\
  \item NetBSD version of AWK programming language, ported
    to other Operating Systems.\\
    \URL{http://mova.org/\~{}cheusov/pub/mk-configure/nbawk/}
  \item Modules support for AWK programming language\\
    \URL{http://sourceforge.net/projects/runawk/}
    \URL{http://pkgsrc.se/lang/runawk/}\\
  \item Tool for distributing tasks over network or CPUs\\
    \URL{http://sourceforge.net/projects/paexec/}
    \URL{http://pkgsrc.se/wip/paexec/}\\
  \end{itemize}
  \end{block}
\end{frame}

%%%%%%%%%%%%%%%%%%%%%%%%%%%%%%%%%%%%%%%%%%%%%%%%%%%%%%%%%%%%%%%%%%%%%%

\begin{frame}[fragile,t]
  \frametitle{MK-CONFIGURE in real world}
  \begin{block}{Real life samples of use}
  \begin{itemize}
  \item Distributed fault tolerant bulk build tool for pkgsrc\\
    \URL{http://mova.org/\~{}cheusov/pub/distbb/}\\
    \URL{http://pkgsrc.se/wip/distbb/}\\
  \item Client/server package search system for pkgsrc\\
    \URL{http://mova.org/\~{}cheusov/pub/pkg\_online/}
    \URL{http://pkgsrc.se/wip/pkg\_online-client/}\\
    \URL{http://pkgsrc.se/wip/pkg\_online-server/}\\
  \item Any project based on traditional
    \ModuleName{bsd.\{prog,lib,subdir\}.mk} can easily be converted
    to \MKC{mk-configure}.
  \end{itemize}
  \end{block}
\end{frame}

%%%%%%%%%%%%%%%%%%%%%%%%%%%%%%%%%%%%%%%%%%%%%%%%%%%%%%%%%%%%%%%%%%%%%%

%%%begin-myprojects

\begin{frame}[fragile,t]
  \frametitle{MK-CONFIGURE in real world}
  \begin{block}{My opensource projects using
      mk-configure (filled hexagon), Mk files (box) and others (oval)}
    \begin{figure}
      \includegraphics[width=\textwidth, height=0.60\textheight, keepaspectratio=false]{my_prjs.eps}
    \end{figure}
%    \begin{figure}
%      \includegraphics[width=0.7\textwidth, height=0.10\textwidth, keepaspectratio=false]{my_prjs2.eps}
%    \end{figure}
  \end{block}
\end{frame}

%%%end-myprojects

%%%%%%%%%%%%%%%%%%%%%%%%%%%%%%%%%%%%%%%%%%%%%%%%%%%%%%%%%%%%%%%%%%%%%%

\begin{frame}[fragile,t]
  \frametitle{MK-C needs your help ;-)}
  \begin{block}{}
  \begin{itemize}
  \item Packagers are welcome (Linux distros, OpenBSD etc.)
  \item MK-C distribution contains \h{a lot of regression tests and
    samples of use} (samples are used for testing too).\\
    \h{Shell accounts} on
    "exotic" UNIX-like platforms are needed (AIX, HP-UX, non-ELF BSDs,
    IRIX, Solaris, Hurd etc.) for testing and development.
  \item Review of the documentation. At the moment only mk-configure(7),
    samples/ and this presentation are available.
  \item sf.net provides two mailing lists:\\
    \h{mk-configure-help} and \h{mk-configure-discuss}.
  \item TODO file in the distribution is full of tasks.
  \end{itemize}
  \end{block}
\end{frame}

%%%%%%%%%%%%%%%%%%%%%%%%%%%%%%%%%%%%%%%%%%%%%%%%%%%%%%%%%%%%%%%%%%%%%%

\begin{frame}[fragile]
  \frametitle{}
  \begin{block}{}
    \begin{center}
      \Huge{The END.}
    \end{center}
  \end{block}
\end{frame}

%%%%%%%%%%%%%%%%%%%%%%%%%%%%%%%%%%%%%%%%%%%%%%%%%%%%%%%%%%%%%%%%%%%%%%

%%%begin-myprojects

\end{document}

%%%end-myprojects
